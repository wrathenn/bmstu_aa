\documentclass[12pt]{report}
\usepackage[utf8]{inputenc}
\usepackage[T2A]{fontenc}
\usepackage[russian]{babel}
%\usepackage[14pt]{extsizes}
\usepackage{listings}
\usepackage{graphicx}
\usepackage{amsmath,amsfonts,amssymb,amsthm,mathtools}
\usepackage{pgfplots}
\usepackage{filecontents}
\usepackage{indentfirst}
\usepackage{eucal}
\usepackage{enumitem}
% Для \abs{}
\usepackage{commath}
\frenchspacing


\usetikzlibrary{datavisualization}
\usetikzlibrary{datavisualization.formats.functions}

\usepackage[left=2cm,right=2cm, top=2cm,bottom=2cm,bindingoffset=0cm]{geometry}
% Для измененных титулов глав:
\usepackage{titlesec, blindtext, color} % подключаем нужные пакеты
\definecolor{gray75}{gray}{0.75} % определяем цвет
\newcommand{\hsp}{\hspace{20pt}} % длина линии в 20pt
% titleformat определяет стиль
\titleformat{\chapter}[hang]{\Huge\bfseries}{\thechapter\hsp\textcolor{gray75}{|}\hsp}{0pt}{\Huge\bfseries}

% plot
\usepackage{xcolor}
\usepackage{stmaryrd}
\usetikzlibrary{datavisualization}
\usetikzlibrary{datavisualization.formats.functions}

\begin{document}
%\def\chaptername{} % убирает "Глава"
    % Титульник
    \thispagestyle{empty}
    \begin{titlepage}
        \noindent \begin{minipage}{0.15\textwidth}
                      \includegraphics[width=\linewidth]{b_logo}
        \end{minipage}
        \noindent\begin{minipage}{0.9\textwidth}
                     \centering
                     \textbf{Министерство науки и высшего образования Российской Федерации}\\
                     \textbf{Федеральное государственное бюджетное образовательное учреждение высшего образования}\\
                     \textbf{~~~«Московский государственный технический университет имени Н.Э.~Баумана}\\
                     \textbf{(национальный исследовательский университет)»}\\
                     \textbf{(МГТУ им. Н.Э.~Баумана)}
        \end{minipage}

        \noindent\rule{18cm}{3pt}
        \newline\newline
        \noindent ФАКУЛЬТЕТ $\underline{\text{«Информатика и системы управления»}}$ \newline\newline
        \noindent КАФЕДРА $\underline{\text{«Программное обеспечение ЭВМ и информационные технологии»}}$\newline\newline\newline\newline\newline


        \begin{center}
            \noindent\begin{minipage}{1.3\textwidth}
                         \centering
                         \Large\textbf{  Отчет по лабораторной работе №1}\newline
                         \textbf{по дисциплине "Анализ алгоритмов"}\newline\newline
            \end{minipage}
        \end{center}

        \noindent\textbf{Тема} $\underline{\text{Расстояние Левенштейна}}$\newline\newline
        \noindent\textbf{Студент} $\underline{\text{Шацкий Р.Е.}}$\newline\newline
        \noindent\textbf{Группа} $\underline{\text{ИУ7-55Б}}$\newline\newline
        \noindent\textbf{Оценка (баллы)} $\underline{\text{~~~~~~~~~~~~~~~~~~~~~~~~~~~}}$\newline\newline
        \noindent\textbf{Преподаватели} $\underline{\text{Волкова Л.Л., Строганов Ю.В.}}$\newline\newline\newline

        \begin{center}
            \vfill
            Москва~---~\the\year
            ~г.
        \end{center}
    \end{titlepage}

    \tableofcontents

    % Введение
    \newpage
    \chapter*{Введение}
    \addcontentsline{toc}{chapter}{Введение}
    \textbf{Расстояние Левенштейна} --- минимальное количество операций вставки одного символа, удаления одного символа
    и замены одного символа на другой, необходимых для преобразования одной строки в другую.

    Расстояние Левенштейна применяется в теории информации и компьютерной лингвистике для:
    \begin{itemize}
        \item Исправления ошибок в слове
        \item Сравнения текстовых файлов утилитой diff
        \item Для сравнения генов, хромосом и белков в биоинформатике
    \end{itemize}

    Цели лабораторной работы:
    \begin{enumerate}
        \item Изучить методы динамического программирования на основе алгоритмов нахождения расстояния
        Левенштейна и Дамерау-Левенштейна
        \item Оценить реализации алгоритмов нахождения расстояния Левенштейна и Дамерау-Левенштейна
    \end{enumerate}

    Задачи лабораторной работы:
    \begin{enumerate}
        \item Изучить принципы работы алгоритмов Левенштейна и Дамерау-Левенштейна
        \item Применить методы динамического программирования для матричной реализации указанных алгоритмов
        \item Получить практические навыки реализации данных алгоритмов: итерационные и рекурсивные версии
        \item Сравнить итерационные и рекурсивные реализации алгоритмов по затрачиваемым ресурсам (время и память)
        \item Получить экспериментальным методом различия во временной эффективности алгоритмов
        \item Описать и обосновать полученные результаты в отчете о выполненной лабораторной работе
        в формате расчетно-пояснительной записки
    \end{enumerate}

    % Аналитическая часть
    \newpage


    \chapter{Аналитическая часть}

    Расстояние Левенштейна \cite{Levenshtein} между двумя строками ---
    минимальное количество операций вставки одного символа, удаления одного символа
    и замены одного символа на другой, необходимых для преобразования одной строки в другую.

    Цена каждой операции может зависеть от вида операции или от участвующих в ней символов, отражающих вероятность
    разных ошибок при вводе текста.
    \\
    Виды операций и их цены:
    \begin{enumerate}
        \item Вставка (insert), $w(a,\lambda)$ --- цена удаления символа $a$
        \item Удаление (delete), $w(\lambda, b)$ --- цена вставки символа $b$
        \item Замена (replace), $w(a, b)$ --- цена замены символа $a$ на символ $b$
    \end{enumerate}

    Для решения задачи о нахождении редакционного расстояния необходимо найти последовательность замен, при которых
    суммарная цена операций будет минимальной. Расстояние Левенштейна - частный случай решения этой задачи
    при заданных условиях:
    \begin{itemize}
        \item $w(a, a) = 0$
        \item $w(a, b) = 1, a \neq b$
        \item $w(a, \lambda) = 1$
        \item $w(\lambda, b) = 1$
    \end{itemize}

    \section{Рекурсивный алгоритм нахождения расстояния Левенштейна}
    В основе вычисления расстояние Левенштейна между двумя строками a и b лежит формула~\ref{eq:D},
    где:
    \begin{itemize}
        \item $\abs{a}$ означает длину строки $a$
        \item $a[i]$ - \emph{i}-ый символ строки $a$
    \end{itemize}

    \begin{equation}
        \label{eq:D}
        D(i, j) = \begin{cases}
                      0 &\text{i = 0, j = 0} \\
                      i &\text{j = 0, i > 0} \\
                      j &\text{i = 0, j >0} \\
                      \min \lbrace \\
                      \qquad D(i, j - 1) + 1 \\
                      \qquad D(i - 1, j) + 1 &\text{i > 0, j > 0} \\
                      \qquad D(i - 1, j - 1) + m(a[i], b[j]) &\text{\ref{eq:m}} \\
                      \rbrace
        \end{cases}
    \end{equation}

    Функция 1.2 позволяет сравнить два символа:
    \begin{equation}
        \label{eq:m}
        m(a, b) = \begin{cases}
                      0 &\text{a = b} \\
                      1 &\text{иначе}
        \end{cases}
    \end{equation}

    Рекурсивный алгоритм реализует формулу~\ref{eq:D}. Логика функции $D$ состоит в следующем:
    \begin{enumerate}
        \item Для получения из пустой строки пустой строки, требуется 0 операций
        \item Для получения из пустой строки строки $b$ требуется \abs{b} операций (все - insert)
        \item Для получения из строки $a$ пустой строки требуется \abs{a} операций (все - delete)
        \item Для получения из строки $a$ строки $b$ требуется выполнить несколько операций.
        Обозначая $a'$ и $b'$ за строки $a$ и $b$ без последнего символа соответсвенно, цену преобразования 
        из строки $a$ в $b$ можно выразить следующим образом:
        \begin{enumerate}
            \item Сумма цены преобразования строки a' в b и цены операции удаления (для преобразования a' в a)
            \item Сумма цены преобразования строки a в b' и цены операции вставки (для преобразования b' в b)
            \item Сумма цены преобразования строки a' в b' и операции замены
            (если a и b оканчиваются на разные символы)
            \item Цена преобразования строки a' в b' (если a и b оканчиваются на одинаковый символ)
        \end{enumerate}
        Минимальная цена преобразования - минимальное значение из приведенных выше вариантов.
    \end{enumerate}

    \section{Итерационный алгоритм нахождения расстояния Левенштейна с использованием матрицы}


\end{document}